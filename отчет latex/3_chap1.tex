\chapter{Теоретические основы OFDM}

\section{Принципы OFDM}

OFDM основан на использовании множества ортогональных поднесущих для передачи данных. Основные операции включают применение обратного быстрого преобразования Фурье (IFFT) на стороне передатчика и прямого быстрого преобразования Фурье (FFT) на стороне приемника, а также добавление циклического префикса для борьбы с межсимвольной интерференцией \cite{oppenheim1999}.

\section{Кодирование и модуляция}

Процесс передачи данных включает кодирование символов, сверточное кодирование, перемежку и модуляцию QPSK, что обеспечивает устойчивость к ошибкам и эффективное использование спектра.

\section{Эффекты канала}

Канал передачи моделируется с учетом многолучевого распространения и аддитивного белого гауссова шума (AWGN), что влияет на качество принимаемого сигнала.